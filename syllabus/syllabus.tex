\documentclass[10pt]{article}

\usepackage[letterpaper, portrait, margin=1.25in]{geometry}

\usepackage{authblk}
\usepackage[yyyymmdd,hhmmss]{datetime}
\usepackage{fancyhdr}
\pagestyle{fancy}
\lhead{ASTR 511: Galaxies as Galaxies}
\rhead{Winter 2021}
\rfoot{\em \tiny Compiled on \today\ at \currenttime}
\cfoot{\thepage}
\lfoot{}


%%%%%%%%%%%%%%%%%%%%%%%%%%%%%%%%%%%%%%%%%%%%%%%%%%%%
%%% author-defined commands
\newcommand\about     {\hbox{$\sim$}}
\newcommand\x         {\hbox{$\times$}}
\newcommand\othername {\hbox{$\dots$}}
\def\eq#1{\begin{equation} #1 \end{equation}}
\def\eqarray#1{\begin{eqnarray} #1 \end{eqnarray}}
\def\eqarraylet#1{\begin{mathletters}\begin{eqnarray} #1 %
                  \end{eqnarray}\end{mathletters}}
\def\non    {\nonumber \\}
\def\DS     {\displaystyle}
\def\E#1{\hbox{$10^{#1}$}}
\def\sub#1{_{\rm #1}}
\def\case#1/#2{\hbox{$\frac{#1}{#2}$}}
\def\about  {\hbox{$\sim$}}
\def\x      {\hbox{$\times$}}
\def\ug               {\hbox{$u-g$}}
\def\gr               {\hbox{$g-r$}}
\def\ri               {\hbox{$r-i$}}
\def\iz               {\hbox{$i-z$}}
\def\a                {\hbox{$a^*$}}
\def\O                {\hbox{$O$}}
\def\E                {\hbox{$E$}}
\def\Oa               {\hbox{$O_a$}}
\def\Ea               {\hbox{$E_a$}}
\def\Jg               {\hbox{$J_g$}}
\def\Fg               {\hbox{$F_g$}}
\def\J                {\hbox{$J$}}
\def\F                {\hbox{$F$}}
\def\N                {\hbox{$N$}}
\def\dd               {\hbox{deg/day}}
\def\mic              {\hbox{$\mu{\rm m}$}}
\def\Mo{\hbox{$M_{\odot}$}}
\def\Lo{\hbox{$L_{\odot}$}}
\def\comm#1           {\tt #1}
\def\refto#1          {\ref #1}
\def\T#1              {({\bf #1})}
\def\H#1              {({\it #1})}

%%%%%%%%%%%%%%%%%%%%%%%%%%%%%%%%%%%%%%%%%%%%%%%%%%%%


\title{ASTR 511: Galaxies as Galaxies}
\author{Mario Juri\'{c} and \v{Z}eljko Ivezi\'c, Department of Astronomy}
\affil{University of Washington, Winter Quarter 2021}
\date{\vspace{-5ex}}

\begin{document}
\maketitle

\vskip 0.3in
\leftline{{\bf  Location and Time:} Tuesday and Thursday 2:00-3:20pm, Virtual (see Slack for Zoom link)}
 
\vskip 0.2in
\leftline{{\bf  Office Hours:} {Friday 4-5pm, by Zoom (same Zoom link as the lectures) }}
\vskip 0.2in
\leftline{{\bf  Grading:} term project 40\%, two homeworks 20\% each, term
                          paper presentation 20\%;}
\leftline{                \hskip 0.8in key: $>$90\%=A, $>$80\%=B, $>$70\%=C, $>$50\%=D.}
\vskip 0.2in
\leftline{{\bf  Class web site:} http://research.majuric.org/public/teaching/astr511/}
\vskip 0.2in
\leftline{{\bf  Required reading:} Ivezi\'{c}, Beers \& Juri\'{c} 2012, ARA\&A, 50, 251.}
\vskip 0.2in
\leftline{{\bf  Reference Books:} \hskip 0.22in   Binney \& Merrifield: {\it Galactic Astronomy}}    
\leftline{   \hskip 1.44in         Binney \& Tremaine: {\it Galactic Dynamics}}
\leftline{   \hskip 1.44in         Reid \& Hawley: {\it New Light on Dark Stars}} 
\leftline{   \hskip 1.44in         Sparke \& Gallagher: {\it Galaxies in the Universe}}
\vskip 0.3in

\leftline{\bf  The main goals for this class are:}
\begin{enumerate}
  \item Introducing the motivation for studying galaxies in general, and the Milky 
            Way in particular. Overview of the most relevant literature. Overview of 
            galaxy formation theories, galaxy dynamics, and connection to dark matter.
  \item Informing students about the current research in galactic astronomy, including discussion of 
            the expected observational and theoretical progress for the next decade. 
  \item Exposing students to practical problems through hands-on seminars, with emphasis
            on numerical methods and data mining, and using modern software engineering tools. 
  \item Training students to give professional talks. 
\end{enumerate} 

\vskip 0.2in
\leftline{\bf  Expected Class Schedule}
\begin{enumerate}
  \item {\bf Tue: Jan 5}   Introduction to the course  (MJ \& ZI) 
%                              - Introduce the syllabus
%                              - Class mailing list
%                              - Have them get github accounts
%                              - Github student developer pack (https://education.github.com/)
%                              - Discuss problem sets
%                                - Mention Anaconda for Python
  \item {\bf Thu: Jan 7}     L1: Review of stellar astrophysics (ZI)
%                              - basic stellar parameters
%                              - open and globular clusters
%                              - simple stellar populations and population synthesis

  \item {\bf Tue: Jan 12}    L2: Basic properties of the Milky Way and the Local Group (ZI)
%                              - the main structural components of the MW
%                              - stars, gas, dust, black hole
%                              - brief overview of metallicity and kinematics distributions
%                              - LMC/SMC
%                              - Sgr dwarf, tidal streams, dwarf satellites
%                              - multi-wavelength view of the Milky Way
  \item {\bf Thu: Jan 14}    L3: Basic properties of galaxies (ZI)
%                              - brief overview of history
%                              - morphological differences and Hubble tuning fork diagram
%                              - exponential vs. deVac intensity profiles
%                              - multi-wavelength views of galaxies
%                              - origins of infrared emission in galaxies
%                              - starburst galaxies and radio galaxies

  \item {\bf Tue: Jan 19}    L4: Luminosity and mass functions of galaxies: I (ZI)
%                              - color-luminosity distributions
%                              - the spatial distribution of galaxies (vs. galaxy properties)
%                              - Butch-Oemler effect (density-morphology relations)
%                              - methods for estimating luminosity and mass functions
%                              - introduction to galaxy formation theories
%                              - galaxy mergers
%                              - dwarf galaxies

  \item {\bf Thu: Jan 21}    L5: Luminosity and mass functions of galaxies: II (ZI)
%                              - color-luminosity distributions
%                              - the spatial distribution of galaxies (vs. galaxy properties)
%                              - Butch-Oemler effect (density-morphology relations)
%                              - methods for estimating luminosity and mass functions
%                              - introduction to galaxy formation theories
%                              - galaxy mergers
%                              - dwarf galaxies

  \item {\bf Tue: Jan 26}   Discussion of homework and term papers (ZI) {\bf (HW 1 due)}
%                              - talk about HW1 - Gaia analysis
%                              - assign papers 

  \item {\bf Thu: Jan 28}    L6: Stellar kinematics in galaxies (MJ)
%                              - introduction to proper motion and radial velocity measurements
%                              - empirical Tully-Fisher and Faber-Jackson relations

  \item {\bf Tue: Feb 2}    L7: Galactic Dynamics I: Potentials and Orbits (MJ)  
%                              - potential theory
%                              - the orbits of stars
%                              - the collisionless Boltzman equation

 \item {\bf Thu: Feb 4}     L8: Galactic Dynamics II: Galaxy models (MJ)
%                              - the Jeans equations
%                              - the virial theorem
%                              - King models

  \item {\bf Tue: Feb 9}     L9: Galactic Dynamics III: Advanced Dynamics (MJ)
%                              - action-angle variables
%                              - distribution functions

  \item {\bf Thu: Feb 11}     L10: Galactic Dynamics IV: Instabilities and Resonances (MJ)
%                              - non-equilibrium behaviors of many-body systems

  \item {\bf Tue: Feb 16}    L11: Evidence for dark matter in galaxies (MJ)
%                              - rotation curves
%                              - velocity dispersion
%                              - weak lensing
%                              - MOND as an alternative hypothesis

  \item {\bf Thu: Feb 18}  Discussion of homework and term project (MJ \& ZI) {\bf (HW 2 due)}

  \item {\bf Tue: Feb 23}  L12: Stellar count distribution in the Milky Way (MJ)
%                              - derive analytic Euclidean counts
%                              - review history of counting stars (Bahcall & Soneira, etc)
%                              - modern SDSS and 2MASS results

  \item {\bf Thu: Feb 25}  L13: Stellar metallicity and kinematics in the Milky Way (ZI) 
%                              - [Fe/H] and [alpha/Fe] as probes of chemical evolution
%                              - SDSS, RAVE and other modern results
%                              - SDSS results for kinematics of stars in the Milky Way
%                              - SDSS constraints for dark matter in the Milky Way disk and halo

  \item {\bf Tue: Mar 2}  L14: Open questions in galactic astronomy and modern sky surveys (ZI) 
%                              - top open questions in galaxy astrophysics
%                              - Gaia, LSST, RAVE, SDSS: science drivers, data products, and data release cadence

  \item {\bf Tue: Mar 4}     Term paper presentations, discussion of term project  
  \item {\bf Thu: Mar 9}     Term paper presentations   
  \item {\bf Thu: Mar 11}   Term paper presentations 
  \item {\bf Tue: Mar 16}   Term project discussion and hackathon 
\end{enumerate}


\vskip 0.2in

\leftline{\bf  Homeworks and term project}

There will be two homeworks, designed as practical warm-up projects to prepare you
for the term project towards the quarter's end. They will be similar from the technical 
point of view: they will involve reading a number of vectors with several million elements 
from provided files (using Python and Jupyter notebooks) followed by simple operations 
with these vectors such as binning, low-order statistics and luminosity function estimation
(ok, this one is not very simple but the code will be provided), and visualization of your results. 

The homework problem description and links to data files will be posted at the class website 
about two weeks before the homework is due (due dates: January 26 for HW1 and February 18 for HW2).

The term project will be based on the recently released Gaia Early Data Release 3 and a few
auxiliary datasets (e.g., ZTF and WISE). More details will be provided during January and
discussed in class on Jan 26. 

We will use modern software engineering tools, such as github for version control and Jupyter 
notebooks, for HW submission and work on the term project. 

\vskip 0.2in

\leftline{\bf Suggested Presentation Papers:} 

Pick one below by {\bf Jan 25} (e-mail your choice to both of us).

\begin{enumerate}
\item Bailer-Jones et al. 2020 (https://arxiv.org/abs/2012.05220):
 {\it Estimating distances from parallaxes. V: Geometric and photogeometric distances to 
1.47 billion stars in Gaia Early Data Release 3} 
\item Belokurov et al. 2018 (MNRAS, 478, 611 ) 
 {\it Co-formation of the disc and the stellar halo}  
\item Gaia Collaboration 2018 (A\&A 616, A11)
{\it Mapping the Milky Way disc kinematics}
\item Gaia Collaboration 2018 (A\&A 616, A10)
{\it Observational Hertzsprung-Russell diagrams}
\item Gaia Collaboration 2018 (A\&A 616, A12)
{\it Kinematics of globular clusters and dwarf galaxies around the Milky Way}
 \item Gaia Collaboration 2019 (A\&A 623, A110)
{\it Variable stars in the colour-absolute magnitude diagram}
\item  Gaia Collaboration: X. Luri et al. 2020 (accepted to A\&A))
{\it Gaia Early Data Release 3: Structure and properties of the Magellanic Clouds}
\item  Gaia Collaboration: T. Antoja et al. 2020 (accepted to A\&A))
{\it Gaia Early Data Release 3: The Galactic anticentre}
\item Helmi, A. \& White, S.D.M. 2001 (MNRAS 323, 529)
         {\it Simple dynamical models of the Sagittarius dwarf galaxy}
\item Johnston et al. 2008 (ApJ, 689, 936)
    {\it Tracing Galaxy Formation with Stellar Halos}
\item  Kauffmann et al. 2003 (MNRAS, 341, 33) 
   {\it Stellar Masses and Star Formation Histories for 80,000 Galaxies from the Sloan Digital Sky Survey} 
\item Kuijken, K. \& Gilmore, G. 1991 (ApJ 367, L9)  
     {\it The galactic disk surface mass density and the Galactic force K(z) at Z = 1.1 kpc}
\item Kuijken, K. \& Tremaine, S. 1994 (ApJ 421, 178)
\item Navarro, J.F., Frenk, C.S. \& White, S.D.M. 1996 (ApJ 462, 563) 
         {\it The Structure of Cold Dark Matter Halos}              
\item Spitzer, L. \& Schwarzschild, M. 1953 (ApJ 118, 106) 
           {\it The Possible Influence of Interstellar Clouds on Stellar Velocities. II}
\item van den Bosch, F.C. \& Dalcanton, J.J. 2000 (ApJ 534, 146)
         {\it Semianalytical Models for the Formation of 
               Disk Galaxies. II. Dark Matter versus Modified Newtonian Dynamics}
\end{enumerate}

These papers are very relevant for our class and cover subjects that are
not going to be discussed in detail. Choose a paper and prepare a 20 min long 
Powerpoint (or equivalent) presentation
(presumably including the most important figures from the paper). Pretend 
you did the work yourself and are giving an invited talk at a meeting. 
There will be about 10-15 min long question and answer session after the talk,
with everyone participating. 
The purpose of this exercise is to 1) learn some science, 2) practice 
extracting relevant information from papers 3) practice giving talks,
4) practice answering questions from your audience.

Do not forget the following good practices: 
1) empty your pockets (no loose change, keys, phone, and such),
2) talk slowly and sufficently loud, 3) don't look at the floor, control
your audience with direct eye contact, don't turn your back to the audience, 
don't be agressive with the pointer, etc. 4) don't rush (don't overload 
your 10 min long presentation), concentrate on the most important points,
5) emphasize what are truly new, and, possibly, unexpected results.
6) comment on the limitations and pitfalls of the presented analysis;
how do we know it's right, could it be wrong?


\end{document}
\item Bahcall, J.N. \& Tremaine, S. 1981 (ApJ 244, 805)
       {\it Methods for determining the masses of spherical systems. I - Test particles around a point mass}
\item Dehnen, W. \& Binney, J.J. 1998 (MNRAS 298, 387)
          {\it Local stellar kinematics from HIPPARCOS data}
\item Ibata, R., et al. 2001 (ApJ 551, 294)
         {\it Great Circle Tidal Streams: Evidence for a Nearly Spherical Massive Dark Halo around the Milky Way}
\item Jaffe, W. 1983 (MNRAS 202, 995) 
          {\it A simple model for the distribution of light in spherical galaxies}
           {\it On the ellipticity of the Galactic disk}
\item Walsh, Willman \& Jerjen 2009 (AJ, 137, 450) 
   {\it The Invisibles: A Detection Algorithm to Trace the Faintest Milky Way Satellites}  
\item McGurk et al. 2010 (AJ, 139, 1261) 
     {\it Principal Component Analysis of SDSS Stellar Spectra}
\item Sesar et al. 2010 (ApJ, 708, 717)
  {\it Light Curve Templates and Galactic Distribution of RR Lyrae Stars from Sloan Digital Sky Survey Stripe 82}
\item Schlaufman et al. 2009 (ApJ, 703, 2177)
  {\it Insight into the Formation of the Milky Way Through Cold Halo Substructure. I. The ECHOS of Milky Way Formation}
\item Law and Majewski 2010 (ApJ, 714, 229) 
 {\it The Sagittarius Dwarf Galaxy: A Model for Evolution in a Triaxial Milky Way Halo}
\item Beers et al. 2012 (ApJ, 746, 34) 
 {\it The Case for the Dual Halo of the Milky Way}
\item Berry et al. 2012 (ApJ, 757, 166)
 {\it The Milky Way Tomography with SDSS. IV. Dissecting Dust}
\item Evans \& Williams 2014 (MNRAS, 443, 791)
 {\it A very simple cusped halo model} 



To-do:
- update http://research.majuric.org/public/teaching/astr511/  to Winter2021. 
- shall we just keep our ARAA review, or add disk and halo from Helmi and ??
- new homeworks:
   1) Gaia EDR3: produce HR diagram and color-code by: counts, pm, apparent magnitude, distance (aML code)
               Stripe 82 or whole sky or NGP? 
     -> perhaps look at selection in https://www.aanda.org/articles/aa/pdf/2019/03/aa33304-18.pdf
   2) simulated WD sample - luminosity function 
   3) Gaia EDR and Stripe 82: calibrate photometric parallax (including FeH), and produce FeH maps, vtan maps 
           search for substructure, 

AGB stars in LMC: G~15-16, see https://www.cosmos.esa.int/web/gaia/edr3-structure-magellanic-clouds

